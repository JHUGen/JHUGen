%
%
\documentclass[aps,superscriptaddress,nofootinbib]{revtex4}
% \documentclass[12pt]{article}
\usepackage{graphicx}
\usepackage{epsfig}
\usepackage{float}
% % \usepackage{cprotect}
%\usepackage{amsmath}

% \textwidth18cm
% \addtolength{\oddsidemargin}{-2cm}
% \addtolength{\evensidemargin}{-2cm}

\begin{document}

\vspace{0.6cm}

\title{ 
\large
Manual for the JHU generator
}
\maketitle
\begin{center}
\small
For simulation of a single-produced resonance at hadron colliders \\
(version 4.7.0, release date August 11, 2014) \\
\normalsize
\end{center}



\noindent
The generator from \cite{Gao:2010qx,Bolognesi:2012,Anderson:2013} is a model-independent generator for studying spin and parity properties of new resonances.  
Please cite \cite{Gao:2010qx,Bolognesi:2012,Anderson:2013} if using the generator.  

The code can be downloaded from \cite{thesite}.  
The generator outputs LHE files which can be passed to parton shower programs for hadronization.  
%The generator purposely does not output sensible cross-sections but rather deals with numbers of events which can be compared for different signal hypotheses.
Only relative values of cross sections are supposed to produce meaningful results, while absolute values are often subject to an arbitrary normalization.

Additionally, the package now includes code for computing the matrix elements standalone which can be used in a numerical matrix element analysis.  

\vspace{0.5cm}
\begin{center}
\line(1,0){250}
\end{center}
\vspace{0.5cm}
\tableofcontents
\begin{center}
%\line(1,0){250}
\end{center}
\vspace{0.5cm}

\newpage

\section{ Installation }

\noindent
Register and download the package from \verb|www.pha.jhu.edu/spin| and untar the file.  Go to the directory \verb|JHUGenerator| where the code exists for generating events with the JHU Generator. In the \verb|makefile|, you have two options for compiler, \verb|Comp = ifort| or \verb|Comp = gfort|.  Then simply compile with:
\begin{verbatim}
$ make
\end{verbatim}

\section{ Configuration }

There are two ways to configure the program, from the command line and in the file \verb|mod_Parameters.F90|.  For documentation from the command line, one can use \verb|JHUGen help|.  In addition, the command line configurables are defined in the file \verb|main.F90|.  When one change the fortran code directly, one should also recompile the code for changes to take effect.  In general, command-line configuration handles general event properties while the configuration file handles all of the couplings and physics handles.

\subsection{ Command line configuration }

The list of command line configurables and the default values are (also defined in the \verb|README|):

\begin{verbatim}
    Collider:   1=LHC, 2=Tevatron, 0=e+e-
    Process:    0=spin-0, 1=spin-1, 2=spin-2 resonance, 50=pp/ee->VH,
                60=weakVBF, 61=pp->Hjj
    MReso:      resonance mass (default=126.00), format: yyy.xx
    DecayMode1: decay mode for vector boson 1 (Z/W+/gamma)
    DecayMode2: decay mode for vector boson 2 (Z/W-/gamma)
                  0=Z->2l,  1=Z->2q, 2=Z->2tau, 3=Z->3nu,
                  4=W->lnu, 5=W->2q, 6=W->taunu,
                  7=gamma, 8=Z->2l+2tau,
                  9=Z->anything, 10=W->lnu+taunu, 11=W->anything
    PChannel:   0=g+g, 1=q+qb, 2=both
    OffXVV:     off-shell option for resonance(X),or vector bosons(VV)
    PDFSet:     1=CTEQ6L1(2001),  2=MSTW(2008),  
                                2xx=MSTW with eigenvector set xx=01..40)
    VegasNc0:   number of evaluations for integrand scan
    VegasNc1:   number of evaluations for accept-reject sampling
    VegasNc2:   number of events for accept-reject sampling
    Unweighted: 0=weighted events, 1=unweighted events
    Interf:     0=neglect interference for 4f final states, 
                1=include interference
    DataFile:   LHE output file
    ReadLHE:    LHE input file from external file (only spin-0)
\end{verbatim}   
   
\noindent
A few more details on some particular parameters:

\begin{itemize}
\item {\verb|VegasNc0,1,2|}: For unweighted event generation VegasNc0 specifies the number of evaluations for the initial integrand scan.  The actual event generation is controlled by either VegasNc1 or VegasNc2. VegasNc1 specifies the number of tries in the accept/reject phase and VegasNc2 is the number of generated events. When generating unweighted events in ReadLHE mode, both VegasNc1 or VegasNc2 can be used to specify the number of generated events. For the generation of weighted events VegasNc1 specifies the number of evaluations for each of 5 iterations during the initial integrand scan. VegasNc2 gives the (approximate) number of generated weighted events.
\item {\verb|OffXVV|}: The program does not work for $ZZ$ or $WW$ if you set them to be on-shell (\verb|OffXVV="000"|) and the mass of the $X$ resonance to be below the $m_{VV}$ threshold.  In general, the more off-shell the process, or the more "1" you have, the less efficient the \verb|VegasNc1| evaluations are.  Specifically, if you are interested then, in producing a resonance with mass below threshold $m_{VV}$ with a very narrow resonance, it is most efficient to generate with \verb|OffXVV="011"|
\item \verb|PChannel|: This parameter is only meaningful in the spin-2 case.  For spin-0, production is possible only via the $gg$ process and for spin-1, production is only possible via the $q\bar{q}$ process.
\item \verb|DecayMode2=7| note: Valid for spin-0 and spin-2, only OffXVV=000 or 100 are possible.  
\item \verb|VegasNc| note: \verb|VegasNc1| allows the user to set the number of tries for the reject/accept method, \verb|VegasNc2| allows the user to let the program run until the number of events is generated.
\end{itemize}

\noindent
Then, as an example of running the generator, you could do:

\noindent
- gg production:
\begin{verbatim}
./JHUGen Collider=1 Process=0 VegasNc2=100000 PChannel=0 OffXVV=011 DecayMode1=0 DecayMode2=0 \\
		Unweighted=.true. DataFile=test1
\end{verbatim}
- ggH $\to$ Zgamma
\begin{verbatim}
./JHUGen DecayMode1=0 DecayMode2=7 OffXVV=010
\end{verbatim}
- VH (both $pp$ and $e^+e^-$ Collider options possible):
\begin{verbatim}
./JHUGen Collider=1 Process=50 Unweighted=1 VegasNc2=100000 OffXVV=011 DataFile=test2 
\end{verbatim}
- VBF:
\begin{verbatim}
./JHUGen Collider=1 Process=60 Unweighted=1 VegasNc2=100000 OffXVV=011 DataFile=test2 
\end{verbatim}
- H+jj:
\begin{verbatim}
./JHUGen Collider=1 Process=61 Unweighted=1 VegasNc2=100000 OffXVV=011 DataFile=test3
\end{verbatim}

%\begin{center}
%\line(1,0){250}
%\end{center}
%\noindent
%\footnotesize
%{\it N.B.}
%There is a beta-version of the generator which has improved efficiency for the generation.  However, it is currently only available for gluon-gluon initiated processes.
%It is by default turned off, but it can be accessed in the file \verb|main.F90|.  The beta version is currently still under validation. 
%\begin{verbatim}
%logical,parameter :: useBetaVersion=.false.
%\end{verbatim}
%\normalsize
%\begin{center}
%\line(1,0){250}
%\end{center}

\noindent
For generating Higgs decay in VBF, H+JJ, or VH production
modes by the JHU generator or NLO gluon fusion with another generator (e.g. POWHEG), use JHU generator in LHE 
conversion mode and specify the decay mode of interest (ZZ, WW, gam gam, Z gam), while the SM fermionic decays 
may be generated by Pythia without loss of generality.

\subsection{ Configuration in parameter file  }

In the file \verb|mod_Parameters.F90|, one does all the configuration of the couplings of the resonance.
After modifying this file, one needs to recompile. 

\subsubsection{ General parameters }

\noindent
Each generation run is different when this is \verb|.true.| 

\begin{verbatim}
seed_random = .true.
\end{verbatim}

\noindent
In the case when \verb|PChannel=2| for a spin-2 resonance, the user can define an approximate ratio of the production of gg and $q\bar{q}$ production.

\begin{verbatim}
fix_channels_ratio = .true.
channels_ratio_fix = 0.25d0    ! desired ratio of
                               ! N_qq/(N_qq+N_gg)
\end{verbatim}


\noindent
For final states with a Z-boson decaying into $f\bar f$, intermediate off-shell photons can be included by switching 
\begin{verbatim}
logical, public, parameter :: includeGammaStar = .false. 
\end{verbatim}
to the value \verb|.true.|. In such case, a lower cut on the photon invariant mass has to be placed in order to avoid the
collinear singularity. This cutoff parameter is controlled by
\begin{verbatim}
real(8),parameter :: MPhotonCutoff = 4d0*GeV.
\end{verbatim}


\noindent
{\it Only for final states with 4 same flavor fermions }, one can include interference effects between the leptons.  The interference is controlled by the command line parameter:
\begin{verbatim}
Interf=0 or 1
\end{verbatim}

\noindent
For the generation of weighted events (command line \verb|Unweighted=0|) an LHE output file is created if
\begin{verbatim}
logical, public, parameter :: writeWeightedLHE = .false. 
\end{verbatim}
is set to \verb|.true.|.


The remaining parameters are more-or-less self-explanatory:
\begin{verbatim}
! we are using units of 100GeV, i.e. Lambda=10 is 1TeV
real(8), public, parameter :: GeV=1d0/100d0
real(8), public, parameter :: percent=1d0/100d0
real(8), public :: M_V,Ga_V
real(8), public, parameter :: M_Z     = 91.1876d0 *GeV      ! Z boson mass (PDG-2011)
real(8), public, parameter :: Ga_Z    = 2.4952d0  *GeV      ! Z boson width(PDG-2011)
real(8), public, parameter :: M_W     = 80.399d0  *GeV      ! W boson mass (PDG-2011)
real(8), public, parameter :: Ga_W    = 2.085d0   *GeV      ! W boson width(PDG-2011)
real(8), public            :: M_Reso  = 125.6d0   *GeV      ! X resonance mass (spin 0, spin 1, spin 2)     (carefule: no longer a parameter, can be overwritten by command line argument)
real(8), public, parameter :: Ga_Reso = 0.00415d0 *GeV      ! X resonance width
real(8), public, parameter :: Lambda  = 1000d0    *GeV      ! Lambda coupling enters in two places
                                                            ! overal scale for x-section and in power
                                                            ! suppressed operators/formfactors (former r).

real(8), public, parameter :: m_el = 0.00051100d0  *GeV         ! electron mass
real(8), public, parameter :: m_mu = 0.10566d0  *GeV              ! muon mass
real(8), public, parameter :: m_tau = 1.7768d0  *GeV                ! tau mass


real(8), public, parameter :: Gf = 1.16639d-5/GeV**2        ! fermi constant
real(8), public, parameter :: vev = 1.0d0/sqrt(Gf*sqrt(2.0d0))
real(8), public, parameter :: gwsq = 4.0d0 * M_W**2/vev**2  ! weak constant squared
real(8), public, parameter :: alpha_QED = 1d0/128.0d0       ! el.magn. coupling
real(8), public, parameter :: alphas = 0.13229060d0         ! strong coupling
real(8), public, parameter :: sitW = dsqrt(0.23119d0)       ! sin(Theta_Weinberg) (PDG-2008)
real(8), public            :: Mu_Fact                       ! pdf factorization scale 
                                                              (set to M_Reso in main.F90)
real(8), public, parameter :: LHC_Energy=14000d0  *GeV       ! LHC hadronic center of mass energy
real(8), public, parameter :: TEV_Energy=1960d0  *GeV       ! Tevatron hadronic center of mass energy
real(8), public, parameter :: ILC_Energy=250d0  *GeV        ! Linear collider center of mass energy
real(8), public, parameter :: POL_A = 0d0                   !e+ polarization. 0: no polarization, 100: 
                                                             helicity = 1, -100: helicity = -1
real(8), public, parameter :: POL_B = 0d0                   !e- polarization. 0: no polarization, 100: 
                                                             helicity = 1, -100: helicity = -1
logical, public, parameter :: H_DK =.true.                  !default to false so H in 
                                                             V > VH (Process = 50) does not decay
real(8), public, parameter :: ptjetcut = 15d0*GeV           ! jet min pt
real(8), public, parameter :: Rjet = 0.5d0                  ! jet deltaR, antikt algorithm 
\end{verbatim}


\noindent
The branching fractions of $Z$ and $W$ bosons depend on the above input parameter and can slightly vary from the given PDG measurements.
Those branchings can be rescaled with the parameters below.
\begin{verbatim}
real(8), public, parameter :: scale_alpha_Z_uu = 1.04282d0 ! scaling factor of alpha (~partial width) for Z > u u~, c c~
real(8), public, parameter :: scale_alpha_Z_dd = 1.04282d0 ! scaling factor of alpha (~partial width) for Z > d d~, s s~, b b~
real(8), public, parameter :: scale_alpha_Z_ll = 1d0       ! scaling factor of alpha (~partial width) for Z > l+ l-
real(8), public, parameter :: scale_alpha_Z_nn = 1d0       ! scaling factor of alpha (~partial width) for Z > nu nu~
real(8), public, parameter :: scale_alpha_W_ud = 1.0993819d0 ! scaling factor of alpha (~partial width) for W > u d, c s
real(8), public, parameter :: scale_alpha_W_ln = 1d0       ! scaling factor of alpha (~partial width) for W > l nu
\end{verbatim}
These default values rescale the branchings to include the NLO QCD corrections ($1+\alpha_s/\pi$).

\subsubsection{ Spin-0 parameters }

\noindent
{\it N.B.  The parameters "ptjetcut" and "Rjet" only apply to Process=60,61.}

\noindent
The \verb|*hg*| parameters control the coupling of a spin-0 resonance to gluons in the production mechanism.  
In practice, the production parameters are not having a large effect since angular corrections from the production mechanism are lost for spinless particles.  
The \verb|*hz*| parameters control the decay into $Z$ and $W$ bosons.  
One has the options to set the spin-0 couplings either from Eq.(9) or Eq.(11) from Ref.~\cite{Bolognesi:2012}.  
To switch between the two, use the parameter \verb|generate_as|.  
%For the parameters in Eq.(9) from Ref.~\cite{Bolognesi:2012}, we now allow for $q^2$ dependent form factors as described in Eq.(3) of~\cite{Anderson:2013} as:
We now allow for $q^2$ dependent form factors as described in Ref.~\cite{Anderson:2013} and given in more detail in the equation below:
\[
g_i(q_1^2,q_2^2) = 
g_i^{\rm SM} 
+ g_i' \frac{\Lambda_i^4}{(\Lambda_i^2 + |q_1^2|)(\Lambda_i^2 + |q_2^2|)} 
~~~~~~~~~~~~~~~~~~~~~~~~~~~~~~~~~~~~~~~~~~~~~~~~~~~~~~~~~~~~~~~~~~~~~~~~~~~~~~
\]
\[
+ g_i^{\prime 2}  \frac{(|q_1^2|+|q_2^2|)}{\Lambda_i^2}
+ g_i^{\prime 3}  \frac{(|q_1^2|-|q_2^2|)}{\Lambda_i^2}
+ g_i^{\prime 4}  \frac{|(q_1+q_2)^2|}{\Lambda_Q^2}
+ g_i^{\prime 5}  \frac{(|q_1^2|^2+|q_2^2|^2)}{\Lambda_i^4}
+ g_i^{\prime 6}  \frac{(|q_1^2|^2-|q_2^2|^2)}{\Lambda_i^4}
+ g_i^{\prime 7}  \frac{|q_1^2| \, |q_2^2|}{\Lambda_i^4}
\]
\noindent
The user has the option to choose between these functional forms, 
where the term multiplying $g_i'$ corresponds to the full functional form and the $g_i''... g_i'''''' $ correspond to an expansion in $\Lambda^2$.
All parameters can be modified in \verb|mod_Parameters.F90| by:
\begin{verbatim}

!-- parameters that define on-shell spin 0 coupling to SM fields, see note
   logical, public, parameter :: generate_as = .false.
   complex(8), public, parameter :: ahg1 = (1.0d0,0d0)
   complex(8), public, parameter :: ahg2 = (0.0d0,0d0)
   complex(8), public, parameter :: ahg3 = (0.0d0,0d0)  ! pseudoscalar
   complex(8), public, parameter :: ahz1 = (1.0d0,0d0)
   complex(8), public, parameter :: ahz2 = (0.0d0,0d0)  ! this coupling does not contribute for gamma+gamma final states
   complex(8), public, parameter :: ahz3 = (0.0d0,0d0)  ! pseudoscalar

!-- parameters that define off-shell spin 0 coupling to SM fields, see note
   complex(8), public, parameter :: ghg2 = (1.0d0,0d0)
   complex(8), public, parameter :: ghg3 = (0.0d0,0d0)
   complex(8), public, parameter :: ghg4 = (0.0d0,0d0)   ! pseudoscalar
   complex(8), public, parameter :: ghz1 = (2.0d0,0d0)
   complex(8), public, parameter :: ghz2 = (0.0d0,0d0)
   complex(8), public, parameter :: ghz3 = (0.0d0,0d0)
   complex(8), public, parameter :: ghz4 = (0.0d0,0d0)   ! pseudoscalar 

!-- parameters that define q^2 dependent form factors
   complex(8), public, parameter :: ghz1_prime = (0.0d0,0d0)
   complex(8), public, parameter :: ghz1_prime2= (0.0d0,0d0)
   complex(8), public, parameter :: ghz1_prime3= (0.0d0,0d0)
   complex(8), public, parameter :: ghz1_prime4= (0.0d0,0d0)
   complex(8), public, parameter :: ghz1_prime5= (0.0d0,0d0)
   complex(8), public, parameter :: ghz1_prime6= (0.0d0,0d0)
   complex(8), public, parameter :: ghz1_prime7= (0.0d0,0d0)

   complex(8), public, parameter :: ghz2_prime = (0.0d0,0d0)
   complex(8), public, parameter :: ghz2_prime2= (0.0d0,0d0)
   complex(8), public, parameter :: ghz2_prime3= (0.0d0,0d0)
   complex(8), public, parameter :: ghz2_prime4= (0.0d0,0d0)
   complex(8), public, parameter :: ghz2_prime5= (0.0d0,0d0)
   complex(8), public, parameter :: ghz2_prime6= (0.0d0,0d0)
   complex(8), public, parameter :: ghz2_prime7= (0.0d0,0d0)

   complex(8), public, parameter :: ghz3_prime = (0.0d0,0d0)
   complex(8), public, parameter :: ghz3_prime2= (0.0d0,0d0)
   complex(8), public, parameter :: ghz3_prime3= (0.0d0,0d0)
   complex(8), public, parameter :: ghz3_prime4= (0.0d0,0d0)
   complex(8), public, parameter :: ghz3_prime5= (0.0d0,0d0)
   complex(8), public, parameter :: ghz3_prime6= (0.0d0,0d0)
   complex(8), public, parameter :: ghz3_prime7= (0.0d0,0d0)

   complex(8), public, parameter :: ghz4_prime = (0.0d0,0d0)
   complex(8), public, parameter :: ghz4_prime2= (0.0d0,0d0)
   complex(8), public, parameter :: ghz4_prime3= (0.0d0,0d0)
   complex(8), public, parameter :: ghz4_prime4= (0.0d0,0d0)
   complex(8), public, parameter :: ghz3_prime5= (0.0d0,0d0)
   complex(8), public, parameter :: ghz3_prime6= (0.0d0,0d0)
   complex(8), public, parameter :: ghz3_prime7= (0.0d0,0d0)

   real(8),    public, parameter :: Lambda_z1 = 10000d0*GeV
   real(8),    public, parameter :: Lambda_z2 = 10000d0*GeV
   real(8),    public, parameter :: Lambda_z3 = 10000d0*GeV
   real(8),    public, parameter :: Lambda_z4 = 10000d0*GeV   
   real(8),    public, parameter :: Lambda_Q  = 10000d0*GeV   
\end{verbatim}

If the switch \verb|includeGammaStar| is set to \verb|.true.| then intermediate off-shell photons are included for $Z$ boson final states. 
Their couplings to the spin-0 resonance are controlled by separate parameters,
\begin{verbatim}
   complex(8), public, parameter :: ghzgs2  = (0.00d0,0d0)
   complex(8), public, parameter :: ghzgs3  = (0.00d0,0d0)
   complex(8), public, parameter :: ghzgs4  = (0.00d0,0d0)
   complex(8), public, parameter :: ghgsgs2 = (0.00d0,0d0)
   complex(8), public, parameter :: ghgsgs3 = (0.00d0,0d0)
   complex(8), public, parameter :: ghgsgs4 = (0.00d0,0d0)
\end{verbatim}
where the first three correspond to $Z\gamma^*$ couplings and the latter three corresponds to $\gamma^* \gamma^*$ interactions.
These two sets of parameters also controll the coupling strength in final states with on-shell photons, i.e. $Z\gamma$ and $\gamma\gamma$.
The anomalous coupling involving the off-shell photon momentum (in $\gamma^* Z$ interactions)
\[
g_1' \frac{ q^2_\gamma }{ \Lambda_2^{Z\gamma} } m_Z^2 \epsilon^*_1 \epsilon^*_2
\]
is set by
\begin{verbatim}
   complex(8), public, parameter :: ghzgs1_prime2= (0.0d0,0d0)
   real(8),    public, parameter :: Lambda_z5 = 10000d0*GeV.
\end{verbatim}

In the weak vector boson fusion process we also allow for different $ZZH$ and $WWH$ couplings. 
Per default, they are assumed to be equal, set by the {\tt ghzi} and {\tt ghzi\_primej} couplings.
If any of the {\tt ghwi} or {\tt ghwi\_primej} couplings is different from zero,
their value will be used instead for the $WWH$ interactions.



\subsubsection{ Spin-1 parameters }

The parameters below represent the couplings given in Eq.~(16) from Ref.~\cite{Bolognesi:2012}. The \verb|*left*| and \verb|*right*| parameters control the production of the spin-1 resonance while the \verb|*_v| and \verb|*_a| parameters control the decay.

\begin{verbatim}
!---parameters that define spin 1 coupling to SM fields, see note
   complex(8), public, parameter :: zprime_qq_left  = (1.0d0,0d0)   !  see note Eq. (4)
   complex(8), public, parameter :: zprime_qq_right = (0.0d0,0d0) 
   complex(8), public, parameter :: zprime_zz_v =  (1.0d0,0d0)!  =1 for JP=1-
   complex(8), public, parameter :: zprime_zz_a =  (0.0d0,0d0)!  =1 for JP=1+
\end{verbatim}   

\subsubsection{ Spin-2 parameters }

\noindent
The \verb|a*| parameters control the coupling of a spin-2 resonance to gluons in the production mechanism.  The \verb|b*| and \verb|c*| parameters control the decay.
One has the options to set the spin-2 couplings either from Eq.(18) or Eq.(19) from Ref.~\cite{Bolognesi:2012}.  To switch between the two, use the parameter \verb|generate_bis|.

\begin{verbatim}
  logical, public, parameter :: generate_bis = .true.
  logical, public, parameter :: use_dynamic_MG = .true. ! .true. (=default), 
													    ! the spin-2 resonance mass with MG^2=(p1+p2)^2, otherwise fixed at M_Reso^2. 

  complex(8), public, parameter :: a1 = (1.0d0,0d0)    ! g1  -- c.f. note
  complex(8), public, parameter :: a2 = (0.0d0,0d0)    ! g2
  complex(8), public, parameter :: a3 = (0.0d0,0d0)    ! g3
  complex(8), public, parameter :: a4 = (0.0d0,0d0)    ! g4
  complex(8), public, parameter :: a5 = (0.0d0,0d0)    ! pseudoscalar, g8

  complex(8), public, parameter :: graviton_qq_left  = (1.0d0,0d0)! graviton coupling to quarks
  complex(8), public, parameter :: graviton_qq_right = (1.0d0,0d0)

  complex(8), public, parameter :: b1 = (1.0d0,0d0)  
  complex(8), public, parameter :: b2 = (0.0d0,0d0)
  complex(8), public, parameter :: b3 = (0.0d0,0d0)
  complex(8), public, parameter :: b4 = (0.0d0,0d0)
  complex(8), public, parameter :: b5 = (0.0d0,0d0)
  complex(8), public, parameter :: b6 = (0.0d0,0d0)
  complex(8), public, parameter :: b7 = (0.0d0,0d0)
  complex(8), public, parameter :: b8 = (0.0d0,0d0)
  complex(8), public, parameter :: b9 = (0.0d0,0d0)  
  complex(8), public, parameter :: b10 =(0.0d0,0d0)  


  complex(8), public, parameter  :: c1 = (1.0d0,0d0)
  complex(8), public, parameter  :: c2 = (0.0d0,0d0)
  complex(8), public, parameter  :: c3 = (0.0d0,0d0)
  complex(8), public, parameter  :: c41= (0.0d0,0d0)
  complex(8), public, parameter  :: c42= (0.0d0,0d0)
  complex(8), public, parameter  :: c5 = (0.0d0,0d0)
  complex(8), public, parameter  :: c6 = (0.0d0,0d0) 
  complex(8), public, parameter  :: c7 = (0.0d0,0d0) 
\end{verbatim}

\section{ Examples }

\noindent
The below examples are not meant to be a complete set, but rather some interesting and relevant cases.  
In many cases, the example is not the only way to produce such a scenario.

\subsection{ $J^P = 0^+_m$ resonance, $X \to ZZ~{\rm or}~WW$}
\label{sec:exA}

\begin{verbatim}
   logical, public, parameter :: generate_as = .true.

   complex(8), public, parameter :: ahg1 = (1.0d0,0d0)
   complex(8), public, parameter :: ahg2 = (0.0d0,0d0)
   complex(8), public, parameter :: ahg3 = (0.0d0,0d0)  ! pseudoscalar
   complex(8), public, parameter :: ahz1 = (1.0d0,0d0)
   complex(8), public, parameter :: ahz2 = (0.0d0,0d0)  
   complex(8), public, parameter :: ahz3 = (0.0d0,0d0)  ! pseudoscalar
\end{verbatim}

\subsection{ $J^P = 0^-_m$ resonance, $X \to ZZ~{\rm or}~WW$}
\label{sec:exB}

\begin{verbatim}
   logical, public, parameter :: generate_as = .true.

   complex(8), public, parameter :: ahg1 = (1.0d0,0d0)
   complex(8), public, parameter :: ahg2 = (0.0d0,0d0)
   complex(8), public, parameter :: ahg3 = (0.0d0,0d0)  ! pseudoscalar
   complex(8), public, parameter :: ahz1 = (0.0d0,0d0)
   complex(8), public, parameter :: ahz2 = (0.0d0,0d0)  
   complex(8), public, parameter :: ahz3 = (1.0d0,0d0)  ! pseudoscalar
\end{verbatim}

\subsection{ $J^P = 0^+_m$ resonance, $X \to \gamma \gamma$}
\label{sec:exC}

In practice, the example $X \to \gamma \gamma$ from this section, Sec.~\ref{sec:exC} and the next Sec.~\ref{sec:exD} are 
kinematically the same but are presented only to illustrate how one takes care of this final state.  

\begin{verbatim}
   logical, public, parameter :: generate_as = .false.

   complex(8), public, parameter :: ghg2 = (1.0d0,0d0)
   complex(8), public, parameter :: ghg3 = (0.0d0,0d0)
   complex(8), public, parameter :: ghg4 = (0.0d0,0d0)   ! pseudoscalar
   complex(8), public, parameter :: ghgsgs2 = (1.0d0,0d0)
   complex(8), public, parameter :: ghgsgs3 = (0.0d0,0d0)
   complex(8), public, parameter :: ghgsgs4 = (0.0d0,0d0)   ! pseudoscalar 
\end{verbatim}

\subsection{ $J^P = 0^-_m$ resonance, $X \to \gamma \gamma$}
\label{sec:exD}

\begin{verbatim}
   logical, public, parameter :: generate_as = .false.

   complex(8), public, parameter :: ghg2 = (1.0d0,0d0)
   complex(8), public, parameter :: ghg3 = (0.0d0,0d0)
   complex(8), public, parameter :: ghg4 = (0.0d0,0d0)   ! pseudoscalar
   complex(8), public, parameter :: ghgsgs2 = (1.0d0,0d0)
   complex(8), public, parameter :: ghgsgs3 = (0.0d0,0d0)
   complex(8), public, parameter :: ghgsgs4 = (0.0d0,0d0)   ! pseudoscalar 
\end{verbatim}

\subsection{ $J^P = 2^+_m$ resonance, $X \to ZZ~{\rm or}~WW~{\rm or}~\gamma \gamma$}

\begin{verbatim}

  complex(8), public, parameter :: a1 = (1.0d0,0d0)    ! g1  -- c.f. draft
  complex(8), public, parameter :: a2 = (0.0d0,0d0)    ! g2
  complex(8), public, parameter :: a3 = (0.0d0,0d0)    ! g3
  complex(8), public, parameter :: a4 = (0.0d0,0d0)    ! g4
  complex(8), public, parameter :: a5 = (0.0d0,0d0)    ! pseudoscalar, g8
  complex(8), public, parameter :: graviton_qq_left  = (1.0d0,0d0)! graviton coupling to quarks
  complex(8), public, parameter :: graviton_qq_right = (1.0d0,0d0)
  
  logical, public, parameter :: generate_bis = .true.
  logical, public, parameter :: use_dynamic_MG = .true.

  complex(8), public, parameter :: b1 = (1.0d0,0d0)
  complex(8), public, parameter :: b2 = (0.0d0,0d0)
  complex(8), public, parameter :: b3 = (0.0d0,0d0)
  complex(8), public, parameter :: b4 = (0.0d0,0d0)
  complex(8), public, parameter :: b5 = (1.0d0,0d0)
  complex(8), public, parameter :: b6 = (0.0d0,0d0)
  complex(8), public, parameter :: b7 = (0.0d0,0d0)
  complex(8), public, parameter :: b8 = (0.0d0,0d0)
  complex(8), public, parameter :: b9 = (0.0d0,0d0)
  complex(8), public, parameter :: b10 =(0.0d0,0d0) 

\end{verbatim}

\subsection{Cross-section calculation and fraction notation}

For a vector boson coupling, we can represent the four independent parameters by two  fractions 
($f_{g2}$ and $f_{g4}$) and two phases ($\phi_{g2}$ and $\phi_{g4}$), defined for the $HZZ$
and $HWW$ couplings as follows (ignoring $g_3$)
%%%%%%%%%%%%%%%%%%%%%%%%%%%%%
%
\begin{eqnarray}
&& f_{gi} =  \frac{|g^{}_{i}|^2\sigma_i}{|g^{}_{1}|^2\sigma_1+|g^{}_{2}|^2\sigma_2+|g^{}_{4}|^2\sigma_4}\,;
~~~~~~~~~
 \phi_{gi} = \arg\left(\frac{g_i}{g_1}\right)\,.
\nonumber
\label{eq:fractions}
\end{eqnarray}
%
%%%%%%%%%%%%%%%%%%%%%%%%%%%%%
In order to obtain the cross-sections $\sigma_i$ corresponding to the $g^{}_{i}=1$ coupling, 
generate large enough (e.g. VegasNc1=1000000, VegasNc2=50000000) number of weighted
({\tt Unweighted=0}) with the corresponding couplings setup ($g^{}_{i}=1$, $g^{}_{j\ne i}=0$).

%%%%%%%%%%%
\section{ JHU Generator Matrix Elements (JHUGenMELA)}

\subsection{Native matrix elements}

After extracting the code, you can go to the directory \verb|JHUGenMELA| to find code for computing matrix elements directly.  
To compile the code, simple do:
\begin{verbatim}
$ make
\end{verbatim}

\noindent
\bf Please take note: The setup is configured for \verb|gfort + gcc version 4.1.2 20080704 (Red Hat 4.1.2-50)| and it is highly dependent on the compiler version.  Please configure for your own setup accordingly.  {\rm (Using \verb|'nm'| command will help decipher the module names you will need)}\\
\\
\rm
\noindent
The usage of the package is straight-forward and an example is given in \verb|testprogram.c|.  
There are 6 main modules allowing both specific production process and production-independent calculation:
\begin{itemize}
\item "modhiggs\_\_evalamp\_gg\_h\_vv": spin-0 matrix elements for $gg$ initiated processes
\item "modzprime\_\_evalamp\_qqb\_zprime\_vv": spin-1 matrix elements for $q\bar{q}$ initiated processes
\item "modgraviton\_\_evalamp\_gg\_g\_vv:" spin-2 matrix elements for $gg$ initiated processes
\item "modgraviton\_\_evalamp\_qqb\_g\_vv": spin-2 matrix elements for $q\bar{q}$ initiated processes
\item "modzprime\_\_evalamp\_zprime\_vv": spin-1 matrix elements production-independent
\item "modgraviton\_\_evalamp\_g\_vv": spin-2 matrix elements production-independent

\end{itemize}

The inputs are the 4-vectors of the incoming patrons and outgoing particles in the CM frame of the object $X$.  
In addition the mass and width of the resonance are required as well as the ID of the outgoing particles.  
Finally the last set of inputs are the couplings themselves.  They are arrays for parameters for a given spin hypothesis
which mirror the parameters configurable in \verb|mod_Parameters.F90|.  
As an example, the arrays are initialized in \verb|testprogram.c|.
% as:
%\begin{verbatim}
%  double Hggcoupl[3][2];
%  double Hvvcoupl[4][2];
%  double Zqqcoupl[2][2];
%  double Zvvcoupl[2][2];
%  double Gqqcoupl[2][2];
%  double Gggcoupl[5][2];
%  double Gvvcoupl[10][2];
%\end{verbatim}

\subsection{Interface with MCFM}

Instructions for setting up the JHUGenMELA with MCFM are in the file \verb|JHUGenMELA/ggZZ_MCFM/README|.

%%%%%%%%%%%

\section{ Release notes }


\noindent
In going from \verb|v4.5.2| to \verb|v4.7.1|, the updates are as follows:

\begin{itemize}
\item more flexibility for $q^2$-dependent form factors
\item separate couplings for ZZH and WWH in weak boson fusion
\item add new process: pp->H+jet (Process=62)
\item JHUGenMELA: extended MCFM ggHZZ matrix elements by anomalous couplings
\item JHUGenMELA: add matrix elements for H+jet and V+H
\end{itemize}



\noindent
In going from \verb|v4.3.2| to \verb|v4.5.2|, the updates are as follows:

\begin{itemize}
\item add an option of intermediate photons for the modes with Z-bosons
\item more flexibility for $q^2$-dependent form factors
\item option of hadronic branching rescaling (NLO QCD corrections) for inclusive decays
\item synchronize JHUGenMELA with the generator and with MCFM library v6.7
\end{itemize}

\noindent
In going from \verb|v4.2.1| to \verb|v4.3.2|, the updates are as follows:

\begin{itemize}
\item update LHE file format and index of partons 
\item improve log printout
\item update ReadLHE mode: $H\to Z\gamma$ output and more flexible input
\item $VH$ production (replaces beta version)
\item more flexibility for $q^2$-dependent form factors
\item tune $q^2$-dependence of couplings for some of the spin-$2_h$ models
\item synchronize JHUGenMELA with the generator
\end{itemize}

\noindent
In going from \verb|v4.0.x| to \verb|v4.2.x|, the updates are as follows:

To JHUGenerator:
\begin{itemize}
\item Fix BR in "all" decay mode
\item Updates to LHE output
\item Option to print out CS\_max, output for g' and Lambdas
\item Introduction of AnalyticMELA for $ee \to ZH$ and $pp \to ZH$ and analytic parton distribution functions
\end{itemize}


\noindent
In going from \verb|v3.1.x| to \verb|v4.0.x|, the updates are as follows:

To JHUGenerator:
\begin{itemize}
\item Addition of VBF and Hjj process channels
\item Possibility to read in VBF LHE event files
\end{itemize}

To JHUGenMELA:
\begin{itemize}
\item Interface with the MCFM program for ggZZ process
\item Matrix elements for VBF and Hjj processes
\end{itemize}


\noindent
In going from \verb|v2.2.6| to \verb|v3.1.8|, the updates are as follows:

To \verb|JHUGenerator|:
\begin{itemize}
\item Capability reading LHE files with Higgs boson production, allows NLO production of spin-0;
\item Extended the list of final state combinations;
\item Log messages, lhe file headers, and minor cleanup.
\item Updates to deal with non-zero lepton masses, lhe file format, and adjust default settings (e.g. lepton interference applied by default and can be configured in command line)
\end{itemize}

To \verb|JHUGenMELA|:
\begin{itemize}
\item Production-independent JHUGenMELA for spin-0, 1, 2;
\item Complex couplings in JHUGenMELA input.
\end{itemize}

\noindent
In going from \verb|v2.2.3| to \verb|v2.2.6|, the updates are as follows:
\begin{itemize}
\item A small fix which corrects the {\it relative fraction} between the $2e2\mu$ and $4e$/$4\mu$ channels when using interference
\item beta version is still under development
\item $q\bar{q} \to$ spin-2 production is more safely performed with settings \verb|PChannel = 2| and $q\bar{q}$ fraction = 1.
\end{itemize}

\noindent
In going from \verb|v2.1.3| to \verb|v2.2.3|, the updates are as follows:
\begin{itemize}
\item Fix interference and randomization in the {\it{beta}} version
\item Add the \verb|JHUGenMELA| modules
\item Small change for compilation on Mac OSX platforms
\item Fix for tau masses in $W$ decays
\end{itemize}

\noindent
In going from \verb|v2.0.2| to \verb|v2.1.x|, the updates are as follows:

\begin{itemize}
\item Histograms are written in file (default: ./data/output.dat) and no longer on the screen.  How to understand the histogram data and how to plot is briefly described in the output.dat file.
\item Added tau masses
\item Added lepton interference in the ZZ4l final state
\item Added switch \verb|generate_as| to choose couplings in spin-0 case (works for on- and off-shell resonance). The default is ".false.".
\item Added the possibility to change graviton-quark couplings. The new parameters are \verb|graviton_qq_left|, \verb|graviton_qq_right| and correspond to $0.5*(1-\gamma^5)$ and $0.5*(1+\gamma^5)$ helicity projectors, respectively. Up to now the coupling was always vector-like. This is also the new default, \verb|graviton_qq_left = graviton_qq_right =1|.
\item The random seed is now fixed with gfortran.
\item The call "./JHUGen help" prints out all available command line options
\item Added new command line option "Unweighted=0 or 1" (default is 1)
\end{itemize}

\clearpage
\appendix

%%%%%%%%%%%%%%%%%%%%%%%%%%%%%%%%%%%%%%%%%%%%%%%%%%%%

\section{Specific configurations}

We define configurations for certain models which are defined in Table~1 of~\cite{Bolognesi:2012}.

%%%-------- 0+
\subsection{"SM-like spin-zero", $0^+$}

\footnotesize
\begin{verbatim}
!-- parameters that define on-shell spin 0 coupling to SM fields, see note
   logical, public, parameter :: generate_as = .false.
   complex(8), public, parameter :: ahg1 = (1.0d0,0d0)
   complex(8), public, parameter :: ahg2 = (0.0d0,0d0)
   complex(8), public, parameter :: ahg3 = (0.0d0,0d0)  ! pseudoscalar
   complex(8), public, parameter :: ahz1 = (1.0d0,0d0)
   complex(8), public, parameter :: ahz2 = (0.0d0,0d0)  ! this coupling does not contribute for gamma+gamma final states
   complex(8), public, parameter :: ahz3 = (0.0d0,0d0)  ! pseudoscalar

!-- parameters that define off-shell spin 0 coupling to SM fields, see note
   complex(8), public, parameter :: ghg2 = (1.0d0,0d0)
   complex(8), public, parameter :: ghg3 = (0.0d0,0d0)
   complex(8), public, parameter :: ghg4 = (0.0d0,0d0)   ! pseudoscalar
   complex(8), public, parameter :: ghz1 = (1.0d0,0d0)
   complex(8), public, parameter :: ghz2 = (0.0d0,0d0)
   complex(8), public, parameter :: ghz3 = (0.0d0,0d0)
   complex(8), public, parameter :: ghz4 = (0.0d0,0d0)   ! pseudoscalar 
\end{verbatim}
\normalsize

%%%-------- 0+h
\subsection{"Higher order spin-zero", $0_h^+$}

\footnotesize
\begin{verbatim}
!-- parameters that define on-shell spin 0 coupling to SM fields, see note
   logical, public, parameter :: generate_as = .false.
   complex(8), public, parameter :: ahg1 = (1.0d0,0d0)
   complex(8), public, parameter :: ahg2 = (0.0d0,0d0)
   complex(8), public, parameter :: ahg3 = (0.0d0,0d0)  ! pseudoscalar
   complex(8), public, parameter :: ahz1 = (1.0d0,0d0)
   complex(8), public, parameter :: ahz2 = (0.0d0,0d0)  ! this coupling does not contribute for gamma+gamma final states
   complex(8), public, parameter :: ahz3 = (0.0d0,0d0)  ! pseudoscalar

!-- parameters that define off-shell spin 0 coupling to SM fields, see note
   complex(8), public, parameter :: ghg2 = (1.0d0,0d0)
   complex(8), public, parameter :: ghg3 = (0.0d0,0d0)
   complex(8), public, parameter :: ghg4 = (0.0d0,0d0)   ! pseudoscalar
   complex(8), public, parameter :: ghz1 = (0.0d0,0d0)
   complex(8), public, parameter :: ghz2 = (1.0d0,0d0)
   complex(8), public, parameter :: ghz3 = (0.0d0,0d0)
   complex(8), public, parameter :: ghz4 = (0.0d0,0d0)   ! pseudoscalar 
\end{verbatim}
\normalsize

%%%-------- 0-
\subsection{"Pseudoscalar spin-zero", $0^-$}

\footnotesize
\begin{verbatim}
!-- parameters that define on-shell spin 0 coupling to SM fields, see note
   logical, public, parameter :: generate_as = .false.
   complex(8), public, parameter :: ahg1 = (1.0d0,0d0)
   complex(8), public, parameter :: ahg2 = (0.0d0,0d0)
   complex(8), public, parameter :: ahg3 = (0.0d0,0d0)  ! pseudoscalar
   complex(8), public, parameter :: ahz1 = (1.0d0,0d0)
   complex(8), public, parameter :: ahz2 = (0.0d0,0d0)  ! this coupling does not contribute for gamma+gamma final states
   complex(8), public, parameter :: ahz3 = (0.0d0,0d0)  ! pseudoscalar

!-- parameters that define off-shell spin 0 coupling to SM fields, see note
   complex(8), public, parameter :: ghg2 = (0.0d0,0d0)
   complex(8), public, parameter :: ghg3 = (0.0d0,0d0)
   complex(8), public, parameter :: ghg4 = (1.0d0,0d0)   ! pseudoscalar
   complex(8), public, parameter :: ghz1 = (0.0d0,0d0)
   complex(8), public, parameter :: ghz2 = (0.0d0,0d0)
   complex(8), public, parameter :: ghz3 = (0.0d0,0d0)
   complex(8), public, parameter :: ghz4 = (1.0d0,0d0)   ! pseudoscalar 
\end{verbatim}
\normalsize

%%%-------- 1-
\subsection{"Vector spin-one", $1^-$}

\footnotesize
\begin{verbatim}
!---parameters that define spin 1 coupling to SM fields, see note
   complex(8), public, parameter :: zprime_qq_left  = (1.0d0,0d0)
   complex(8), public, parameter :: zprime_qq_right = (0.0d0,0d0)
   complex(8), public, parameter :: zprime_zz_v =  (1.0d0,0d0)!  =1 for JP=1-
   complex(8), public, parameter :: zprime_zz_a =  (0.0d0,0d0)!  =1 for JP=1+
\end{verbatim}
\normalsize

%%%-------- 1+
\subsection{"Pseudovector spin-one", $1^+$}

\footnotesize
\begin{verbatim}
!---parameters that define spin 1 coupling to SM fields, see note
   complex(8), public, parameter :: zprime_qq_left  = (1.0d0,0d0)
   complex(8), public, parameter :: zprime_qq_right = (0.0d0,0d0)
   complex(8), public, parameter :: zprime_zz_v =  (0.0d0,0d0)!  =1 for JP=1-
   complex(8), public, parameter :: zprime_zz_a =  (1.0d0,0d0)!  =1 for JP=1+
\end{verbatim}
\normalsize

%%%-------- 2+
\subsection{"Minimal Graviton, spin-two", $2^+$}

N.B. If an exclusive production mode is desired (e.g. $q\bar{q}$ or $gg$), this is handled at command-line configuration level via the \verb|PChannel| variable.

\footnotesize
\begin{verbatim}
!-- parameters that define spin 2 coupling to SM fields, see note
! minimal coupling corresponds to a1 = b1 = b5 = 1 everything else 0
  complex(8), public, parameter :: a1 = (1.0d0,0d0)    ! g1  -- c.f. draft
  complex(8), public, parameter :: a2 = (0.0d0,0d0)    ! g2
  complex(8), public, parameter :: a3 = (0.0d0,0d0)    ! g3
  complex(8), public, parameter :: a4 = (0.0d0,0d0)    ! g4
  complex(8), public, parameter :: a5 = (0.0d0,0d0)    ! pseudoscalar, g8
  complex(8), public, parameter :: graviton_qq_left  = (1.0d0,0d0)! graviton coupling to quarks
  complex(8), public, parameter :: graviton_qq_right = (1.0d0,0d0)

!-- see mod_Graviton
  logical, public, parameter :: generate_bis = .true.
  logical, public, parameter :: use_dynamic_MG = .true.

  complex(8), public, parameter :: b1 = (1.0d0,0d0)    !  all b' below are g's in the draft
  complex(8), public, parameter :: b2 = (0.0d0,0d0)
  complex(8), public, parameter :: b3 = (0.0d0,0d0)
  complex(8), public, parameter :: b4 = (0.0d0,0d0)
  complex(8), public, parameter :: b5 = (1.0d0,0d0)
  complex(8), public, parameter :: b6 = (0.0d0,0d0)
  complex(8), public, parameter :: b7 = (0.0d0,0d0)
  complex(8), public, parameter :: b8 = (0.0d0,0d0)
  complex(8), public, parameter :: b9 = (0.0d0,0d0)
  complex(8), public, parameter :: b10 =(0.0d0,0d0)  ! this coupling does not contribute for gamma+gamma final states


  complex(8), public, parameter  :: c1 = (1.0d0,0d0)
  complex(8), public, parameter  :: c2 = (0.0d0,0d0)
  complex(8), public, parameter  :: c3 = (0.0d0,0d0)
  complex(8), public, parameter  :: c41= (0.0d0,0d0)
  complex(8), public, parameter  :: c42= (0.0d0,0d0)
  complex(8), public, parameter  :: c5 = (0.0d0,0d0)
  complex(8), public, parameter  :: c6 = (0.0d0,0d0)
  complex(8), public, parameter  :: c7 = (0.0d0,0d0)
\end{verbatim}
\normalsize

%%%-------- 2+h
\subsection{"Higher order Graviton, spin-two", $2^+_h$}

\footnotesize
\begin{verbatim}
!-- parameters that define spin 2 coupling to SM fields, see note
! minimal coupling corresponds to a1 = b1 = b5 = 1 everything else 0
  complex(8), public, parameter :: a1 = (0.0d0,0d0)    ! g1  -- c.f. draft
  complex(8), public, parameter :: a2 = (0.0d0,0d0)    ! g2
  complex(8), public, parameter :: a3 = (0.0d0,0d0)    ! g3
  complex(8), public, parameter :: a4 = (1.0d0,0d0)    ! g4
  complex(8), public, parameter :: a5 = (0.0d0,0d0)    ! pseudoscalar, g8
  complex(8), public, parameter :: graviton_qq_left  = (1.0d0,0d0)! graviton coupling to quarks
  complex(8), public, parameter :: graviton_qq_right = (1.0d0,0d0)

!-- see mod_Graviton
  logical, public, parameter :: generate_bis = .true.
  logical, public, parameter :: use_dynamic_MG = .true.

  complex(8), public, parameter :: b1 = (0.0d0,0d0)    !  all b' below are g's in the draft
  complex(8), public, parameter :: b2 = (0.0d0,0d0)
  complex(8), public, parameter :: b3 = (0.0d0,0d0)
  complex(8), public, parameter :: b4 = (1.0d0,0d0)
  complex(8), public, parameter :: b5 = (0.0d0,0d0)
  complex(8), public, parameter :: b6 = (0.0d0,0d0)
  complex(8), public, parameter :: b7 = (0.0d0,0d0)
  complex(8), public, parameter :: b8 = (0.0d0,0d0)
  complex(8), public, parameter :: b9 = (0.0d0,0d0)
  complex(8), public, parameter :: b10 =(0.0d0,0d0)  ! this coupling does not contribute for gamma+gamma final states


  complex(8), public, parameter  :: c1 = (1.0d0,0d0)
  complex(8), public, parameter  :: c2 = (0.0d0,0d0)
  complex(8), public, parameter  :: c3 = (0.0d0,0d0)
  complex(8), public, parameter  :: c41= (0.0d0,0d0)
  complex(8), public, parameter  :: c42= (0.0d0,0d0)
  complex(8), public, parameter  :: c5 = (0.0d0,0d0)
  complex(8), public, parameter  :: c6 = (0.0d0,0d0)
  complex(8), public, parameter  :: c7 = (0.0d0,0d0)
\end{verbatim}
\normalsize

%%%-------- 2-h
\subsection{"Higher order Graviton, spin-two", $2^-_h$}

\footnotesize
\begin{verbatim}
!-- parameters that define spin 2 coupling to SM fields, see note
! minimal coupling corresponds to a1 = b1 = b5 = 1 everything else 0
  complex(8), public, parameter :: a1 = (0.0d0,0d0)    ! g1  -- c.f. draft
  complex(8), public, parameter :: a2 = (0.0d0,0d0)    ! g2
  complex(8), public, parameter :: a3 = (0.0d0,0d0)    ! g3
  complex(8), public, parameter :: a4 = (0.0d0,0d0)    ! g4
  complex(8), public, parameter :: a5 = (1.0d0,0d0)    ! pseudoscalar, g8
  complex(8), public, parameter :: graviton_qq_left  = (1.0d0,0d0)! graviton coupling to quarks
  complex(8), public, parameter :: graviton_qq_right = (1.0d0,0d0)

!-- see mod_Graviton
  logical, public, parameter :: generate_bis = .true.
  logical, public, parameter :: use_dynamic_MG = .true.

  complex(8), public, parameter :: b1 = (0.0d0,0d0)    !  all b' below are g's in the draft
  complex(8), public, parameter :: b2 = (0.0d0,0d0)
  complex(8), public, parameter :: b3 = (0.0d0,0d0)
  complex(8), public, parameter :: b4 = (0.0d0,0d0)
  complex(8), public, parameter :: b5 = (0.0d0,0d0)
  complex(8), public, parameter :: b6 = (0.0d0,0d0)
  complex(8), public, parameter :: b7 = (0.0d0,0d0)
  complex(8), public, parameter :: b8 = (1.0d0,0d0)
  complex(8), public, parameter :: b9 = (0.0d0,0d0)
  complex(8), public, parameter :: b10 =(0.0d0,0d0)  ! this coupling does not contribute for gamma+gamma final states


  complex(8), public, parameter  :: c1 = (1.0d0,0d0)
  complex(8), public, parameter  :: c2 = (0.0d0,0d0)
  complex(8), public, parameter  :: c3 = (0.0d0,0d0)
  complex(8), public, parameter  :: c41= (0.0d0,0d0)
  complex(8), public, parameter  :: c42= (0.0d0,0d0)
  complex(8), public, parameter  :: c5 = (0.0d0,0d0)
  complex(8), public, parameter  :: c6 = (0.0d0,0d0)
  complex(8), public, parameter  :: c7 = (0.0d0,0d0)
\end{verbatim}
\normalsize


\begin{thebibliography}{99}
\bibitem{Gao:2010qx}
Y.Y. Gao, A. V. Gritsan, Z.J. Guo, K. Melnikov, M. Schulze and N. V. Tran, "Spin-Determination of Single-Produced Resonances at Hadron Colliders". Phys. Rev. D {\bf 81}, 075022 (2010). arXiv:1001.3396 [hep-ph].
\bibitem{Bolognesi:2012}
S. Bolognesi, Y.Y. Gao, A. V. Gritsan, K. Melnikov, M. Schulze, N. V. Tran and A. Whitbeck, "On the Spin and Parity of Single-Produced Resonance at the LHC". Phys. Rev. D {\bf 86}, 095031 (2012). arXiv:1208.4018 [hep-ph].
\bibitem{Anderson:2013}
I. Anderson, S. Bolognesi, F. Caola, Y.Y. Gao, A. V. Gritsan, C. B. Martin, K. Melnikov, M. Schulze, N. V. Tran, A. Whitbeck, Y. Zhou, "Constraining anomalous HVV interactions at proton and lepton colliders". Phys. Rev. D {\bf 89}, 035007 (2014). arXiv:1309.4819 [hep-ph].
\bibitem{thesite}
See webpage: www.pha.jhu.edu/spin
\end{thebibliography}

\end{document}
